\chapter{Conclusion}

Congratulations! You've now made a cable-pulley system, controlled its
dimensions in ways that are meaningful, and made it progressively more complex.
We invested a fair bit of effort on the front end of this design (e.g. using 3-D
sketches rather than 2-D sketches) to enable feature re-use down the line.

This progression, from planar pulleys, to tilted pulleys, to pulleys with fleet
angle, is one I commonly go through in the course of my designs. If I'm lucky, I
can eventually progress back to planar pulleys, which among other things keeps
the machinists happy.

There are a few CAD methods used in this tutorial that I'd like to call out once
more.

\begin{enumerate}
\item{} By deleting relations rather than deleting sketch entities, we can avoid
rebuilding the Feature Tree.
\item{} The more I design, the more I separate sketches that define geometry from
sketches that are used in features. I do this for two reasons:


\begin{itemize}
\item{} These geometry sketches won't be absorbed into the features, making them
  easier to find and modify later.
\item{} Multiple features are influenced by a system's geometry. For instance, the
  cable diameter affects both the cable sweep and the pulley revolve. It is
  best to define these dimensions in a single location, so that when they
  change (and they inevitably will), you only have to change them in one
  place.
\end{itemize}
\item{} Using 3D sketches requires a few unintuitive techniques, but you should be
able to constrain them in any way you can imagine. Don't be afraid to make
copious use of \kode{Planes}. They can easily be deleted later if they're not
useful.
\item{} CAD the cables. You'll thank yourself later.
\end{enumerate}

I hope this tutorial shortens your learning curve. Please reach out to me if you
have feedback or questions. Godspeed!

\hfill\break

-- Ben Smith

February 2022
