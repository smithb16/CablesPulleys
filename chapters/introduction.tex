\chapter{Introduction}

Every time I design a cable and pulley system, I struggle through the same
questions. How does a 3D sketch work? Do I need a 3D sketch? Should I make the cables first or the pulleys first?

Sometimes I decide ``I don't actually need to CAD the cables.'' Time saved by
not cadding the cables is dwarfed by the time spent modifying hardware later, say
when the cables run directly through another part.

\section{Our Notation}

Throughout this tutorial, we'll use the following notation:

\begin{itemize}
\item{} \emph{Italics} is used to denote sketch and feature names, such as \emph{Line 1} and
  \emph{Pulley 1}. While SolidWorks may apply a name like ``Line1@Sketch2'', we'll ignore those names and rename them ourselves.
\item{} \kode{Green code} is used to notate SolidWorks features that you should click, such as \relation{Extrude}, \relation{Circle}, and \relation{Concentric}.
\item{} \texttt{Typewriter text} is used to denote keystrokes, such as \texttt{Tab}
  and \texttt{R}. Here, \texttt{R} refers to pressing the \texttt{R} key alone,
  not in a capitalized form.

\texttt{Shift+R} refers to ``Hold \texttt{Shift} and press \texttt{R}''.
\item{} Constraints will be described as below, which reads ``Add a
    \relation{Coincident}
  relation between \emph{Line 1} and \emph{Line 2}.''
\end{itemize}

\begin{center}
\begin{tabular}{ccc}
  \hline
  \relation{Coincident} & \emph{Line 1} & \emph{Line 2} \\
  \hline
\end{tabular}
\end{center}

\begin{itemize}
  \item{} Removing constraints will be described as below, which reads ``Remove
    the existing \relation{Coincident} relation between \emph{Line 1} and
    \emph{Line 2}.''
\end{itemize}

\begin{center}
\begin{tabular}{ccc}
  \hline
  \xrelation{Coincident} & \emph{\sout{Line 1}} & \emph{\sout{Line 2}} \\
  \hline
\end{tabular}
\end{center}

\section{Modifying the Pace}

\label{sec:modifying_the_pace}
This tutorial is designed for those with basic SolidWorks skills. I only explain
the concepts relevant to 3D sketches and the relevant features. For those who
find the pace too slow, each section is summarized with a \textbf{Section Recap}.
These steps, along with the images and a few mental leaps, should keep the pace
appropriate.

\section{Overview}

\label{sec:overview}

In this tutorial, we'll make a pulley/cable system, utilizing SolidWorks 3D
sketching capabilities. We will start by sketching a system of planar pulleys (Section~\ref{sec:planar-pulleys}), turning our
sketches into a multi-body part (Section~\ref{sec:making_solid_bodies}). We will
then progressively make the system more complicated, moving the system off the
cardinal plane (Section~\ref{sec:non_orthogonal_pulleys}) and making the cables non-tangential to the pulleys (Section~\ref{sec:non-tangential-pulleys}).
